%%%%%%%%%%%%%%%%%%%%%%%%%%%%%%%%%%%%%%%%%
% Medium Length Professional CV
% LaTeX Template
% Version 2.0 (8/5/13)
%
% This template has been downloaded from:
% http://www.LaTeXTemplates.com
%
% Original author:
% Trey Hunner (http://www.treyhunner.com/)
%
% Important note:
% This template requires the resume.cls file to be in the same directory as the
% .tex file. The resume.cls file provides the resume style used for structuring the
% document.
%
%%%%%%%%%%%%%%%%%%%%%%%%%%%%%%%%%%%%%%%%%

%----------------------------------------------------------------------------------------
%	PACKAGES AND OTHER DOCUMENT CONFIGURATIONS
%----------------------------------------------------------------------------------------

\documentclass{resume} % Use the custom resume.cls style

\usepackage[T2A]{fontenc}
\usepackage[utf8]{inputenc}
\usepackage[russian]{babel}
\usepackage{xcolor}% http://ctan.org/pkg/xcolor
\usepackage{hyperref}% http://ctan.org/pkg/hyperref
\usepackage{fontawesome}% http://ctan.org/pkg/fontawesome
\hypersetup{
  colorlinks=true,
  linkcolor=blue!50!red,
  urlcolor=blue
}

\name{Chekanov Arseny} % Your name
\address{Moscow, Russia} % Your address
\address{
    \faMobile \enspace +7(925)6183765 \\ 
    \faEnvelopeO \enspace \href{mailto:senya@chekanov.net}{senya@chekanov.net} \\ 
    \faGithub \enspace \href{https://github.com/SenchoPens}{github.com/SenchoPens}
}
\begin{document}

%----------------------------------------------------------------------------------------
%	EDUCATION SECTION
%----------------------------------------------------------------------------------------

\begin{rSection}{EDUCATION}
    \begin{rSubsubsection}{Higher School of Economics, Moscow}{Sep 2020 -- Expected Jun 2024}{Bachelor in Applied Mathematics and Computer Science}{}
    \end{rSubsubsection}
\end{rSection}

%----------------------------------------------------------------------------------------
%	TECHNICAL STRENGTHS SECTION
%----------------------------------------------------------------------------------------

\begin{rSection}{Skills}

    \begin{tabular}{ @{} >{\bfseries}l @{\hspace{6ex}} l}
        Programming Languages & \textbf{C++, Python, Golang}, Haskell, Bash, Nix, C, Assembly \\
        Technologies          & \textbf{MongoDB}, Redis, \textbf{Docker}, Protobuf, \textbf{Linux}, NixOS, \textbf{REST API}, SQL, \\
                              & Hyperledger Fabric, \textbf{Postman}, BDD \\
        Tools                 & \textbf{Git}, GitHub, CI/CD, Vim, \textbf{Swagger/OpenAPI}\\
        Human Languages       & \textbf{English: Advanced} (\href{https://github.com/hyperledger/fabric-docs-i18n/pull/80}{worked} as a translator), Russian: Native\\
        Mathematics           & \textbf{Linear Algebra} (Regressions, SVD), Graph Theory, \textbf{Calculus}, \\
                              & Abstract Algebra, \textbf{Statistics}, \textbf{Probability Theory}, Topology
    \end{tabular}

\end{rSection}

%----------------------------------------------------------------------------------------
%	INTERESTS SECTION
%----------------------------------------------------------------------------------------

\begin{rSection}{INTERESTS}
    \begin{tabular}{ @{} >{}l @{\hspace{6ex}} l}
        Functional and declarative programming (\textbf{Haskell}, Nix)  & \textbf{DevOps} and \textbf{Linux Administration} \\
        \textbf{Backend} development and \textbf{Parallel programming}         & \textbf{Data Processing} and \textbf{Data Science} \\
    \end{tabular}
\end{rSection}

%----------------------------------------------------------------------------------------
%	WORK EXPERIENCE SECTION
%----------------------------------------------------------------------------------------

\begin{rSection}{WORK EXPERIENCE}
    \begin{rSubsection}{Software Development Intern}{July -- September 2021}{\href{https://procsy.ru}{JSC ProCSy} (\href{https://infotecs.ru/}{InfoTeCS} group)}{Moscow, Russia}
    \item Developed in a team a legal-tech \textbf{blockchain Golang web service}, implemented \textbf{JWT} auth, wrote \textbf{MongoDB} queries, \textbf{Swagger/OpenAPI} docs and \textbf{Ginkgo BDD} \& \textbf{Postman} tests, configured \textbf{Docker} deployment.
    \end{rSubsection}
    \begin{rSubsection}{Software Developer}{May 2018}{«WOT Express» media}{Moscow, Russia}
    \item Developed a \textbf{Python} Telegram bot that forwarded posts to Telegram with \textbf{Vk \& Telegram API}.
    \end{rSubsection}
    \begin{rSubsection}{Software Developer}{May 2017 -- Oct 2017}{Law Firm <<Stimulus>>}{Moscow, Russia}
    \item Developed a \textbf{Python} Telegram bot to query real estate register by using it's \textbf{API} and \textbf{Google Maps API} to create geo links. I had to work alone, but I brought the project to completion from ground-up.
    \end{rSubsection}
\end{rSection}

%----------------------------------------------------------------------------------------
%	Personal Projects SECTION
%----------------------------------------------------------------------------------------

\begin{rSection}{Personal Projects}
    \begin{rSubsection}{\href{https://gitlab.com/shipr1505/backend}{Shipper}}{Apr 2020}{}{}
    \item Developed in a team a \textbf{Dockerized microservice Golang API} with inter-service \textbf{Protobuf} messaging, \textbf{JWT} auth and social network \textbf{API} integration, wrote efficient complex \textbf{MongoDB} pipelines.
    \end{rSubsection}

    \begin{rSubsection}{\href{https://gitlab.com/SenchoPens/hatapp}{Hat game}}{May 2019}{}{}
    \item Implemented an \textbf{Android app} using \textbf{Python} with \textbf{NLP} analysis of Russian National Corpus in a week.
    \end{rSubsection}

    \begin{rSubsection}{\href{https://github.com/SenchoPens/cloud-bees}{Cloud Bees}}{Nov 2017 -- Mar 2018}{}{}
    \item Developed a \textbf{distributed CRDT database} using \textbf{Cloud Haskell} with \textbf{elliptic-curve encryption}.
    \item Wrote an \href{https://github.com/mreluzeon/block-monad}{article} about developing distributed applications with \textbf{Cloud Haskell}.
    \end{rSubsection}

    \begin{rSubsection}{\href{https://github.com/SenchoPens/bioinformatic}{Searching new D-segments}}{Jun 2017}{}{}
    \item Analyzed 3 Gb of bioinformatics data with \textbf{Python} and \textbf{Matplotlib} to find 2 new D-segments.
    \end{rSubsection}

    \begin{rSubsection}{Finding tautologies in text}{Feb -- Mar 2017}{}{}
    \item Used \textbf{word2vec}, \textbf{k-means} and \textbf{ROC curves} to create an algorithm for finding tautologies in text.
    \end{rSubsection}
\end{rSection}

%----------------------------------------------------------------------------------------
%	ACHIEVEMENTS SECTION
%----------------------------------------------------------------------------------------

\begin{rSection}{Achievements}
    \begin{rSubsection}
        {All-Russian Olympiad in informatics <<Lomonosov>>}{May 2020}
        {\href{https://drive.google.com/file/d/1IZ79oAVpzXzu7v7-jUK-btdeVrDk8Che/view?usp=sharing}{Gold medalist}}{}
    \item \textbf{Top 2\% of 189 participants}. Solved CS tasks with \textbf{C++} using \textbf{advanced data structures and algorithms}.
    \end{rSubsection}

    \begin{rSubsection}
        {«Digital Transformation of the \textbf{Charity} Sector» IBM \& HSE Summer School}{Jul 2019}
        {\href{https://www.hse.ru/en/news/298686356.html}{Winner}}{}
    \item Developed a \textbf{Python} \href{https://github.com/SenchoPens/samumoskva_bot}{Telegram bot} using \textbf{SQLite} for an international \textbf{charity} organization SAMU Social.
    \end{rSubsection}
\end{rSection}

\begin{rSection}{Voluntary Work}
    \begin{rSubsection}{Contributor of open-source projects}{June 2020 -- preset}{}{}
    \item I contribute to open source projects either by writing
        \href{https://github.com/darkkeks/kks/pull/125/files}{code} or
        \href{https://github.com/hyperledger/fabric-docs-i18n/pull/80}{documentation},
        \href{https://github.com/search?q=is\%3Aopen+is\%3Aissue+author\%3ASenchoPens+archived\%3Afalse&type=}{opening issues},
        or \href{https://opencollective.com/guest-d7d943a1}{donating money}.
    \end{rSubsection}
    \begin{rSubsection}{Teacher at a math club}{Sep 2018 -- May 2019}{}{}
    \item As a high-school student, teached extracurricular math to fifth- and sixth-graders.
    \end{rSubsection}
\end{rSection}

%----------------------------------------------------------------------------------------
%	EXAMPLE SECTION
%----------------------------------------------------------------------------------------

%\begin{rSection}{Section Name}

%Section content\ldots

%\end{rSection}

%----------------------------------------------------------------------------------------

\end{document}
